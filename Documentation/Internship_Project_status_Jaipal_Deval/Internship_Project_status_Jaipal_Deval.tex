\documentclass{article}
\usepackage{amsmath}
\usepackage{graphicx}
\usepackage{geometry}
\usepackage[hidelinks]{hyperref}
\geometry{
	a4paper,
	total={210mm,290mm},
	left=20mm,
	right=20mm,
	top=20mm,
	bottom=20mm}
\begin{document}
\title{Intership Project status Report}
\maketitle
\section{Personal Details}
\textbf{Name:} Jaipal deval\\

\hspace{-5mm}\textbf{Email Id:} jaipaldeval@gmail.com\\

\hspace{-5mm}\textbf{Mobile Number:} +918884722900\\

\hspace{-5mm}\textbf{Title of Project:} Humanoid Robot\\

\hspace{-5mm}\textbf{Duration of the internship:} 26 May 2014 to 10 July 2014\\

\hspace{-5mm}\textbf{Summary of your contribution to the project:} The project assigned to us was Humanoid Robot. We decided to split the project into two parts mechanical design and electronic design. I took care of the mechanical aspect of the robot. The mechanical design was further categorized in three sections i.e. Design\&Construction, Stability and Motion. The design and construction part was handled by both of us were we both brainstorm the faults in the existing design and came up with the new design and build the whole structure together. The stability part was handled by me, where the C.G of the structure was calculated which helped in calculation of servo angles and in motion section I helped in describing what servo sequence would make the robot balance on one leg or make robot walk.
\section{Project Status Report}
\textbf{Objective of the work:}My interest lies in automation, robot is the best example of automation. Among the list of projects provided to us humanoid robot deals greatly with automation so humanoid robot was the obvious choice\\

\hspace{-5mm}\textbf{Scope of the work:}Build a humanoid robot with basic modes of locomotion\\ 

\hspace{-5mm}\textbf{Completion:}The calculation of the C.G,sequence of servo for motion of the robot and design\&construction of the robot \\

\hspace{-5mm}\textbf{Results and Discussion:}We were successful in building a 16 DOF humanoid which can walk, balance on one leg, do sit-ups. We had a choice of working on existing structure or build our own design from Scratch. We went with choice of working with the existing structure because it would save time, but we found many faults in the existing structure so we discarded that structure. Built our own structure simultaneously interfaced 16 servos on the microcontroller and calculated the C.G of the structure and then robot was programmed. Stability and motion of the robot was handled by me which involved calculation of C.G (centre of gravity), describing the sequence of servos for motion of the robot both were completed successfully.          \\

\hspace{-5mm}\textbf{Features \& Bugs:}The features of the robot are, walk is humanlike, can balance on one leg, can perform sit-ups. Drawbacks are, existing structure cannot accommodate a battery on its body, and this can be fixed with little modification of the structure or the physical modification of the battery. Cannot walk on uneven surface  \newpage
\hspace{-5mm}\textbf{Future Work:}
\begin{enumerate}
	\item Making modules for different possible movements.
	\item Interfacing gyros for auto stabilization of the robot.
	\item Building a GUI for controlling the robot autonomously.
	\item Interfacing camera along with a smart processor like raspberry pi or a beagle board
	\item Interfacing various sensors like ultrasonic sensors for distance calculation, accelerometer,Xigbee for wireless communication.
\end{enumerate}
 \begin{thebibliography}{99}
 	\bibitem{zero} Zero-moment point method for stable biped walking\\
 	DCT no.: 2009.072\\
 	Internship report\\
 	Eindhoven, July 2009.\\
 	
 	\bibitem{real} Real Time Biped Walking Gait Pattern Generator for a Real Robot\\
 	Department of Computer Science and Technology,\\
 	University of Science and Technology of China,\\
 	Hefei, 230026, China\\
 	
 	\bibitem{design} Design and Implementation of a Simplified Humanoid Robot with 8 DOF\\
 	Department of Electronics and Communication Engineering,\\ Hindustan Institute of
 	Technology and Science,\\ Hindustan University, Chennai, India\\
 	
 	\bibitem{kin} Kinematics and Inverse Kinematics for the Humanoid Robot\\
 	Rowland O’Flahertyy, Peter Vieiray, Michael Greyy,\\
 	Paul Ohz, Aaron Bobicky, Magnus Egerstedty, and Mike Stilmany\\
 	
 	\bibitem{kinematics} Kinematics and Dynamics of a New 16 DOF Humanoid Biped Robot with Active Toe Joint\\
 	Regular Paper\\
 	C. Hernández-Santos, E. Rodriguez-Leal1, R. Soto1 and J.L. Gordillo\\
 	
 	\newpage
 	\bibitem{robotic} Robotics (12 of Addis Ababa Lectures): Forward Kinematics of Spatial Robots\\
 	\url{https://www.youtube.com/watch?v=NXWzk1toze4}\\
 	
 	\bibitem{robotic1}Robotics (1/3 of INGOU Lectures): An Introduction and Kinematic Configuration\\
 	\url{https://www.youtube.com/watch?v=gNhK4VoV9P8}\\
 	
 	\bibitem{robotics2}Robotics (2/3 of IGNOU Lectures): Kinematics--Denavit and Hartenberg Parameters\\
 	\url{https://www.youtube.com/watch?v=CnWTUsVle2A}\\
 	
 	\bibitem{robotics3}Robotics (3/3 of IGNOU Lectures): Forward and Inverse Kinematic Analyses\\
 	\url{https://www.youtube.com/watch?v=duKD8cvtBTI}\\
 	
 	\bibitem{roboticswalk}Robot Basic Walk Tutorial\\
 	\url{https://www.youtube.com/watch?v=Xhz6m6fu494}\\
 	
 \end{thebibliography}
 
\end{document}

